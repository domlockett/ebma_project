\documentclass{beamer}
\usetheme[]{Boadilla} 

\definecolor{WUSTLgreen} {RGB} {44, 80, 54}
\definecolor{WUSTLred} {RGB} {149, 1, 1}
\definecolor{WUSTLtan} {RGB} {229, 210, 184}
\setbeamercolor{structure}{fg=WUSTLred, bg=WUSTLgreen}
\setbeamercolor{normaltext}{bg=black, fg=WUSTLtan}
\setbeamercolor{block title}{fg=WUSTLred}
\setbeamercolor{block title example}{fg=WUSTLgreen}
\setbeamercolor{frametitle}{fg=WUSTLgreen}
\setbeamercolor{title}{fg=WUSTLgreen}



\setbeamertemplate{blocks}[rounded]

\usepackage{graphicx}
\usepackage{amsmath}
\usepackage{multirow, multicol}
\usepackage{mathpazo}
\usepackage{amsthm}
\usepackage{amssymb}
\usepackage{setspace}
\usepackage{hyperref}
\usepackage{array,colortbl,booktabs}
\usepackage{soul}

\newcommand{\bframe}{\begin{frame}}
\newcommand{\jacob}{\end{frame}}
\newcommand{\bi}{\begin{itemize}}
\newcommand{\ei}{\end{itemize}}
\newcommand{\be}{\begin{enumerate}}
\newcommand{\ee}{\end{enumerate}}
\newcommand{\vp}{\vspace{.5cm}}
\newcommand{\bis}{\begin{itemize}[<+->]}
\newcommand{\bes}{\begin{enumerate}[<+->]}
\newcommand{\red}[1]{\color{WUSTLred}{#1}}
\newcommand{\bc}{\begin{center}}
\newcommand{\ec}{\end{center}}
\newcommand{\eb}{\end{block}}
\newcommand{\ds}{\displaystyle}

\newcommand{\skipslide}{\addtocounter{framenumber}{-1}}


\newcommand{\ykm}{ $\mathbf{y}_{k-1,j}$}
\newcommand{\ykmg}{ \mathbf{y}_{k-1,j}}



\usepackage{amssymb}
\usepackage{natbib}
\usepackage{tabularx}

\title[CAT \\for Public Opinion Surveys]{Computerized Adaptive Testing \\for Public Opinion Surveys}
\author[~  Montgomery (WUSTL) and Cutler (Duke) ]{Jacob Montgomery \\ \textit{Washington University in Saint Louis} \\ {\tiny and} \\  Josh Cutler \\ \textit{Duke University}}
\date[SLAMM! 2012]{2012 Saint Louis Area Methods Meeting}


\begin{document}
%\frame{\titlepage}

\frame{

\bc
\Large
 \textbf{\red Computerized Adaptive testing \\for Public Opinion Surveys}\\


\vp


\begin{columns}[t] 
\footnotesize
\column{.6in} 
\column{2in} 
\bc
\textbf{Jacob M. Montgomery}\\ 
Dept. of Political Science\\ 
Washington University in St. Louis\\
jacob.montgomery@wustl.edu 
\ec
\column{.25in}
\column{1.5in} 
\bc
\textbf{Joshua Cutler}\\ 
Dept. of Political Science\\ 
Duke University\\
josh.cutler@gmail.com
\ec
\column{1in} 
\end{columns}
\ec



}





\section{Introduction}

\frame{
\frametitle{Talk overview}
 \bi
    \item \textbf{What and why}
   \item Algorithm details
    \item Simulations

    \item Empirical application: political knowledge
   \item Limitations and future directions
    \item Survey software
  \ei
}



\end{document}


\frame
{
  \frametitle{What is CAT?}
  
\bis
    \item Technique used in educational testing (e.g. GRE)
    \item Used to measure latent traits ($\theta$)
    \item Use \textit{previous responses} to select best \textit{next question}
      \bi
        \item ex., No benefit in asking ``high'' scoring individuals
          easy questions
        \item Identifying John Roberts $\rightarrow$ you can identify Joseph Biden 
     \ei
    \item Analogous to branching % Josh: Be ready to explain why we don't just use branching
\ei
}

\frame
{
  \frametitle{Why should we use CAT?}
  
  \bis
    \item Long batteries  raise costs and increase non-response
   \item  Many resort to reduced scales
    \item CAT provides:
    \bi 
      \item increased precision
      \item increased accuracy
    \ei
  \item OK $\ldots$ prove it!
  \ei
}


% \frame{
% \skipslide
% \frametitle{Talk overview}
%  \bi
%     \item What and why
%    \item \textbf{Algorithm details}
%    \item Simulations
%    \item Survey software
%     \item Empirical application: political knowledge
%    \item Limitations and future directions
%   \ei
% }

\section{Algorithm}

\frame
{
  \frametitle{Algorithm overview}
\bc
      \includegraphics[scale=.6]{CAT_Diagram}
\ec
}



\frame{
  \frametitle{Algorithm details}
  
\begin{block}{3PL Logit}
  $$ p_i(\theta_j) \equiv Pr(y_{ij}=1 | \theta_j) = c_i + (1-c_i) \frac{exp\big(Da_i(\theta_j-b_i) \big)}{1+ exp\big(Da_i(\theta_j-b_i)\big)}$$
 \eb

\vp

\pause
\begin{block}{Likelihood function}
$$ L(\theta_j| \mathbf{y}_{k-1}) = \prod_{i=1}^{k-1} p_i(\theta_j)^{y_{ij}} q_i(\theta_j)^{(1-y_{ij})}$$
\eb


\vp
Comments:
\bi
\item $q_i(\theta_j) = 1-p_i(\theta_j)$, 
\item D=1 for logistic, D=1.702 for ``probit''
\item $a_i$, $b_i$, and $c_i$ are given
\ei


}

\frame
{
\frametitle{Algorithm details: Ability estimation}
  
\begin{block}{Expected \textit{a posteriori} (EAP) estimate}
$$\hat{\theta}^{(EAP)}_j \equiv E(\theta_j | \mathbf{y}_{k-1})= \frac{\int \theta_j \pi(\theta) L(\theta|\mathbf{y}_{k-1})d\theta_j}{\int \pi(\theta) L(\theta|\mathbf{y}_{k-1}) d\theta_j}$$
\eb

\vp
\vp

Comments:
\bis
\item $\pi(\theta) \sim N(\mu_{\theta}, \frac{1}{\sqrt{\tau_\theta}})$
\item Sensitive to prior selection: $\pi(\theta) \sim N(0, 1.75)$
\item Estimated via Gauss-Hermite quadrature
\item Common alternative is MAP
\ei

}



\frame
{
\frametitle{Algorithm details: Item selection}
  
\begin{block}{Minimum expected posterior variance (MEPV)}
$$\begin{array}{l}
P(y^*_{ik}=1| \ykmg) \pause Var(\theta_j |\ykmg, y^*_{ik}=1) \pause+  \\
~~~~~P(y^*_{ik}=0|\ykmg) Var(\theta_j | \ykmg, y^*_{ik}=0)
\end{array}$$
\eb
\vp
\pause
\begin{block}{Posterior variance}
$$ Var(\theta_j|\ykmg, y^*_{ik})=\pause \frac{\int (\theta_j-\hat{\theta}_j^{(EAP)*})^2 \pi(\theta)
   L(\theta_j|\mathbf{y}_{k-1}, y^*_{kj})d\theta_j}{\int \pi(\theta)
   L(\theta_j|\mathbf{y}_{k-1}, y^*_{kj}) d\theta_j}$$
\eb


\pause
Comments:
\bis
\item \textit{Many} alternatives: including MFI, MEI, MLWI, MEPWI \\ (van der Linden 1998)
\item (Relatively) expensive
\item Favors items:
\bi
\item high discrimination
\item close\textit{ish} difficulty parameter
\ei
\ei

}


\frame{
\frametitle{Algorithm details: Stopping}

\bis
\item Fixed-length: $n>n_{max}$
\item Variable-length: $1/Var(\theta_j) > \tau_{stop}$
\ei

}



% \frame
% {
%   \frametitle{Algorithm details}
  
% \begin{block}{3PL Logit}
%   $$ p_i(\theta_j) \equiv Pr(y_{ij}=1 | \theta_j) = c_i + (1-c_i) \frac{exp\big(Da_i(\theta-b_i) \big)}{1+ exp\big(Da_i(\theta-b_i)\big)}$$
%  \eb

% \vp

% \pause
% \begin{block}{Likelihood function}
% $$ L(\theta_j| \mathbf{y}_{k-1}) = \prod_{i=1}^{k-1} p_i(\theta_j)^{y_{ij}} q_i(\theta_j)^{(1-y_{ij})}$$
% \eb

% }

% \frame
% {
% \frametitle{Algorithm details: Ability estimation}
  
% \begin{block}{Expected \textit{a posteriori} (EAP) estimate}
% $$\hat{\theta}^{(EAP)}_j \equiv E(\theta_j | \mathbf{y}_{k-1})= \frac{\int \theta_j \pi(\theta) L(\theta|\mathbf{y}_{k-1})d\theta_j}{\int \pi(\theta) L(\theta|\mathbf{y}_{k-1}) d\theta_j}$$
% \eb

% \vp
% \vp
% \pause
% Comments:
% \bis
% \item $a_i$, $b_i$, and $c_i$ are given
% \item Sensitive to prior selection: $\pi(\theta) \sim N(0, 1.75)$
% \item Estimated via Gauss-Hermite quadrature
% \ei

% }



% \frame
% {
% \frametitle{Algorithm details: Item selection}
  
% \begin{block}{Minimum expected posterior variance (MEPV)}
% $$\begin{array}{l}
% P(y^*_{ik}=1| \ykmg) \pause Var(\theta_j |\ykmg, y^*_{ik}=1) \pause+  \\
% ~~~~~P(y^*_{ik}=0|\ykmg) Var(\theta_j | \ykmg, y^*_{ik}=0)
% \end{array}$$
% \eb

% \vp
% \vp
% \pause
% Comments:
% \bis
% \item \textit{Many} alternatives: including MFI, MEI, MLWI, MEPWI \\ (van der Linden 1998)
% \item (Relatively) expensive
% \item Favors items:
% \bi
% \item high discrimination
% \item close\textit{ish} difficulty parameter
% \ei
% \ei

% }

% \frame{
% \frametitle{Algorithm details: Stopping}

% \bis
% \item Fixed-length: $n>n_{max}$
% \item Variable-length: $1/Var(\theta_j) > \tau_{stop}$
% \ei

% }

\section{Simulations}

\frame{
\skipslide
\frametitle{Talk overview}
 \bi
    \item What and why
   \item Algorithm details
   \item \textbf{Simulations}
   \item Empirical application: political knowledge
   \item Limitations and future directions
   \item Survey software
  \ei
}


\frame{
\frametitle{Item characteristic curve (ICC) and posterior: Static scale }
\bc
\includegraphics[scale=.45]{ExFixedSlides1}
\ec
}

\frame{ \skipslide
\frametitle{Static scale}
\bc
\includegraphics[scale=.45]{ExFixedSlides2}
\ec
}


\frame{ \skipslide
\frametitle{Static scale}
\bc
\includegraphics[scale=.45]{ExFixedSlides3}
\ec
}


\frame{ \skipslide
\frametitle{Static scale}
\bc
\includegraphics[scale=.45]{ExFixedSlides4}
\ec
}

\frame{ \skipslide
\frametitle{Static scale}
\bc
\includegraphics[scale=.45]{ExFixedSlides5}
\ec
}

\frame{
\frametitle{Dynamic scale}
\bc
\includegraphics[scale=.45]{ExDynSlides1}
\ec
}


\frame{ \skipslide
\frametitle{Dynamic scale}
\bc
\includegraphics[scale=.45]{ExDynSlides2}
\ec
}

\frame{ \skipslide
\frametitle{Dynamic scale}
\bc
\includegraphics[scale=.45]{ExDynSlides3}
\ec
}


\frame{ \skipslide
\frametitle{Dynamic scale}
\bc
\includegraphics[scale=.45]{ExDynSlides4}
\ec
}

\frame{ \skipslide
\frametitle{Dynamic scale}
\bc
\includegraphics[scale=.45]{ExDynSlides5}
\ec
}



\frame{
%\frametitle{Fix me}


\includegraphics[scale=.58]{MSE3SLIDES}


}


\section{Empirics}

\frame{
\skipslide
\frametitle{Talk overview}
 \bi
    \item What and why
   \item Algorithm details
   \item Simulations
   \item \textbf{Empirical application: political knowledge}
   \item Limitations and future directions
   \item Survey software
  \ei
}


\frame{
  \frametitle{Application: Political knowledge}
  
  \bis
    \item Pilot survey: 38  political knowledge questions
    % \bi
    %  \item Widely used 
    %  \item Widely criticized
    %  \ei
    \item Calibrated model with  AMT respondents (n=604)
    \item Compared static and CAT scales with fresh sample (n=204)
     \begin{itemize}
        \item 101 assigned static 10-item battery
        \item 103 assigned CAT 10-item  battery
        \item Remaining 28 items administered to both groups
      \end{itemize}
  \end{itemize}
}

\frame
{
  \frametitle{Empirical Results}

  \bis  
    \item Computed ``real'' $\theta$ using all 38 responses
    \item Computed Mean Squared Error $(\theta_j - \hat{\theta_j})^2 + Var(\hat{\theta_j})$ 
  \ei
}

\frame
{
%  \frametitle{Fix me}
 \includegraphics[scale=.58]{MSESurveySlides}
}


\frame
{
  \frametitle{Empirical Results}
  \skipslide
  \begin{itemize}
    \item Computed ``real'' $\theta$ using all 38 responses
    \item Computed Mean Squared Error $(\theta_j - \hat{\theta_j})^2 + Var(\hat{\theta_j})$ 
    \item When n is small, CAT outperforms a static battery
    \pause
    \item Wilcoxon tests confirm
  \end{itemize}
}


\section{Conclusion}

\frame{
\frametitle{Limitations}

\bi
\item Measures traits only
\item Calibration required
\item Question order
\item Unstable calibration
\ei

}



\frame
{
  \frametitle{Where to go from here?}
  
  \bi
    \item Improve knowledge questions
    \item Larger sample
\pause
  \item Informative priors
\pause
    \item Polytomous responses
   \item Big-5 100 item inventory
\pause
    \item Investigate different selection criteria 
  \ei
}


\frame
{
  \frametitle{Wrap up}
  
CAT provides:
\bis
\item More accuracy
\item More precision
\item Same low price
\ei

\pause
\vp

We showed:
\bis
\item How CAT works
\item Why it works better (simulations)
\item Empirical pilot study
\ei



\pause
\vp

Next:
\bis
\item Software
\ei
% \bc
% \large{\red Thanks!}
% \ec

}



\section{Software}




\frame
{

  \frametitle{Our needs}
  \bis
    \item Flexibility
    \item Modularity
    \item Speed 
    \item Burstability
  \ei
}

\frame
{

  \frametitle{Our Solution}
  \bis
    \item Hosted in the Cloud (Heroku.com)
    \item R Webservice using Rook
    \item No MCMC right now, numerical integration
    \item Ruby on Rails front end
  \ei
}

\frame
{
%  \frametitle{Architecture Diagram}
%  \begin{columns}
 %   \column{.65\textwidth}
 %   \begin{figure}
\bc
      \includegraphics[scale=.51]{Survey_Diagram}
\ec
%    \end{figure}
    
 %   \column{.35\textwidth}
  %  { \small
   %   \begin{enumerate}
    %    \item RoR sends response pattern to Rook
     %   \item Rook returns next best question with $\theta$ estimates
      %  \item Unless stopping rule is reached, the question is administered
       % \item The response is stored in the MYSQL DB
       % \item Back to step 1
      %\end{enumerate}
    %}
 % \end{columns}
  
}

%  \frame
%  {
%    \frametitle{Demo}
%  \bc
% \href{www.fixthis.com}{Take the quiz!}
% \ec
%  }






\appendix
\newcounter{finalframe}
\setcounter{finalframe}{\value{framenumber}}

%\section{Extra math}



\section{Example knowledge questions}



\frame
{
  \frametitle{Static Battery - 1}
  \begin{enumerate}
    \item The ability of a minority of senators to prevent a vote on a bill is known as 
  	\begin{itemize} 
  		\item suspension of the rules
  		\item enrollment
  		\item a veto
  		\item a fillibuster
  	\end{itemize} 
  	\item The President may NOT 
  	\begin{itemize} 
  		\item declare war
  		\item pardon criminals without justification
  		\item appoint federal officials when Congress is in recess
  		\item refuse to sign legislation passed by Congress
  	\end{itemize}
 \end{enumerate}
}

\frame
{
  \frametitle{Static Battery - 2}
  \begin{enumerate}
 	\item How many senators are elected from each state? 
  	\begin{itemize} 
  		\item It depends on the population of the state
  		\item Four
  		\item Two
  		\item One
  	\end{itemize}
  	\item How long is one term for a member of the U.S. Senate? 
  	\begin{itemize} 
  		\item Eight years
  		\item Six years
  		\item Four years
  		\item Two years
  	\end{itemize}
 \end{enumerate}
}

\frame
{
  \frametitle{Static Battery - 3}
  \begin{enumerate}
 	\item The Secretary of State 
  	\begin{itemize} 
  		\item serves a two-year term
  		\item serves the state governments
  		\item is nominated by the president
  		\item heads the armed services
  	\end{itemize}
  	\item The Byrd Rule is relevant 
  	\begin{itemize} 
  		\item during the confirmation of cabinet members
  		\item for national party conventions
  		\item during appropriations hearings
  		\item for the reconciliation process
  	\end{itemize}
 \end{enumerate}
}

\frame
{
  \frametitle{Static Battery - 4}
  \begin{enumerate}
 	 \item The NRA is an organization that advocates for 
  	\begin{itemize} 
  		\item election reform
  		\item a cleaner environment
  		\item the rights of gun owners
  		\item abortion rights
  	\end{itemize}
  	\item A President may serve 
  	\begin{itemize} 
  		\item any number of terms
  		\item three terms
  		\item two terms
  		\item one term
  	\end{itemize}
 \end{enumerate}
}

\frame
{
  \frametitle{Static Battery - 5}
  \begin{enumerate}
 	\item On which of the following federal programs is the most money spent each year? 
  	\begin{itemize} 
  		\item Education
  		\item Subsidies to farmers and agriculture
  		\item Medicare
  		\item Aid to foreign countries
  	\end{itemize}
  	\item The U.S. House of Representatives has how many voting members? 
  	\begin{itemize} 
  		\item 441
  		\item 435
  		\item 200
  		\item 100
  	\end{itemize}
  \end{enumerate}
}


\section{Item-level parameters}

\frame
{
  \frametitle{Item Level Parameter Estimates - 1}
  
  \begin{table}[scale=.5]
    \label{cal-params}
    \begin{center}
      \footnotesize
      \begin{tabular}{rrrrr}
        Item & Guessing $(c_i)$& Difficulty $(b_i)$ & Discrimination $(a_i)$ & $P(x=1|\theta_j=0)$ \\ 
        1 & 0.50 & 0.19 & 3.11 & 0.68 \\ 
        2& 0.00 & -1.05 & 0.73 & 0.68 \\ 
        \dag3& 0.25 & 0.49 & 3.74 & 0.35 \\ 
        4& 0.23 & -0.35 & 0.84 & 0.67 \\ 
        5 & 0.11 & 0.55 & 1.31 & 0.41 \\ 
        \dag6 & 0.06 & -0.77 & 2.32 & 0.86 \\ 
        7& 0.00 & -2.26 & 2.61 & 1.00 \\ 
        \dag8 & 0.00 & -2.08 & 2.72 & 1.00 \\ 
        \dag9 & 0.23 & -0.91 & 2.19 & 0.91 \\ 
        10 & 0.00 & -2.64 & 1.66 & 0.99 \\ 
        11 & 0.50 & 0.68 & 2.32 & 0.59 \\ 
        12 & 0.31 & 0.49 & 2.85 & 0.45 \\ 
        13 & 0.09 & 2.20 & 2.06 & 0.10 \\ 
        14 & 0.34 & 0.67 & 1.92 & 0.49 \\ 
        15& 0.00 & 2.29 & 0.27 & 0.35 \\ 
      \end{tabular}
    \end{center}
    \footnotesize
    $n=604$.  \dag Item included in 10-item fixed scale.
  \end{table}
}

\frame
{
  \frametitle{Item Level Parameter Estimates - 2}
  
  \begin{table}[scale=.5]
    \label{cal-params}
    \begin{center}
      \footnotesize
      \begin{tabular}{rrrrr}
        Item & Guessing $(c_i)$& Difficulty $(b_i)$ & Discrimination $(a_i)$ & $P(x=1|\theta_j=0)$ \\ 
        \dag16 & 0.11 & 0.56 & 2.56 & 0.28 \\ 
        17 & 0.14 & -1.35 & 0.79 & 0.78 \\ 
        18 & 0.01 & -1.46 & 0.66 & 0.72 \\ 
        \dag19 & 0.37 & 1.03 & 2.94 & 0.40 \\ 
        20 & 0.00 & -1.63 & 0.49 & 0.69 \\ 
        \dag21 & 0.18 & 1.88 & 4.47 & 0.18 \\ 
        22 & 0.00 & -2.56 & 0.69 & 0.85 \\ 
        23& 0.17 & 0.68 & 2.43 & 0.30 \\ 
        24 & 0.24 & 0.39 & 2.39 & 0.45 \\ 
        25 & 0.50 & -1.14 & 2.19 & 0.96 \\ 
        26 & 0.00 & -1.31 & 1.15 & 0.82 \\ 
        27 & 0.33 & -0.61 & 2.24 & 0.86 \\ 
        28 & 0.00 & -2.43 & 1.06 & 0.93 \\ 
        \dag29 & 0.19 & -1.54 & 2.27 & 0.98 \\ 
        30 & 0.23 & 0.25 & 1.11 & 0.56 \\ 
      \end{tabular}
    \end{center}
    \footnotesize
    $n=604$.  \dag Item included in 10-item fixed scale.
  \end{table}
}

\frame
{
  \frametitle{Item Level Parameter Estimates - 3}
  
  \begin{table}[scale=.5]
    \label{cal-params}
    \begin{center}
      \footnotesize
      \begin{tabular}{rrrrr}
        Item & Guessing $(c_i)$& Difficulty $(b_i)$ & Discrimination $(a_i)$ & $P(x=1|\theta_j=0)$ \\ 
        31 & 0.22 & -1.06 & 2.03 & 0.92 \\ 
        32 & 0.23 & -0.41 & 1.78 & 0.75 \\ 
        33 & 0.50 & -1.36 & 1.83 & 0.96 \\ 
        \dag34 & 0.34 & -0.02 & 2.95 & 0.68 \\ 
        35 & 0.10 & -1.90 & 2.04 & 0.98 \\ 
        \dag36 & 0.23 & 0.20 & 1.63 & 0.56 \\ 
        37 & 0.15 & 0.23 & 1.27 & 0.51 \\ 
        38 & 0.00 & -2.58 & 1.21 & 0.96 \\ 
      \end{tabular}
    \end{center}
    \footnotesize
    $n=604$.  \dag Item included in 10-item fixed scale.
  \end{table}
}


\setcounter{framenumber}{\value{finalframe}}

\end{document}